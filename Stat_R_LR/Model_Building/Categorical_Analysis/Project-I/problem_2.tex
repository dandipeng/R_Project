\documentclass[]{article}
\usepackage{lmodern}
\usepackage{amssymb,amsmath}
\usepackage{ifxetex,ifluatex}
\usepackage{fixltx2e} % provides \textsubscript
\ifnum 0\ifxetex 1\fi\ifluatex 1\fi=0 % if pdftex
  \usepackage[T1]{fontenc}
  \usepackage[utf8]{inputenc}
\else % if luatex or xelatex
  \ifxetex
    \usepackage{mathspec}
  \else
    \usepackage{fontspec}
  \fi
  \defaultfontfeatures{Ligatures=TeX,Scale=MatchLowercase}
\fi
% use upquote if available, for straight quotes in verbatim environments
\IfFileExists{upquote.sty}{\usepackage{upquote}}{}
% use microtype if available
\IfFileExists{microtype.sty}{%
\usepackage{microtype}
\UseMicrotypeSet[protrusion]{basicmath} % disable protrusion for tt fonts
}{}
\usepackage[margin=1in]{geometry}
\usepackage{hyperref}
\hypersetup{unicode=true,
            pdftitle={STA 138 Project Problem 2},
            pdfauthor={Dandi Peng, Yuhan Ning},
            pdfborder={0 0 0},
            breaklinks=true}
\urlstyle{same}  % don't use monospace font for urls
\usepackage{color}
\usepackage{fancyvrb}
\newcommand{\VerbBar}{|}
\newcommand{\VERB}{\Verb[commandchars=\\\{\}]}
\DefineVerbatimEnvironment{Highlighting}{Verbatim}{commandchars=\\\{\}}
% Add ',fontsize=\small' for more characters per line
\usepackage{framed}
\definecolor{shadecolor}{RGB}{248,248,248}
\newenvironment{Shaded}{\begin{snugshade}}{\end{snugshade}}
\newcommand{\KeywordTok}[1]{\textcolor[rgb]{0.13,0.29,0.53}{\textbf{#1}}}
\newcommand{\DataTypeTok}[1]{\textcolor[rgb]{0.13,0.29,0.53}{#1}}
\newcommand{\DecValTok}[1]{\textcolor[rgb]{0.00,0.00,0.81}{#1}}
\newcommand{\BaseNTok}[1]{\textcolor[rgb]{0.00,0.00,0.81}{#1}}
\newcommand{\FloatTok}[1]{\textcolor[rgb]{0.00,0.00,0.81}{#1}}
\newcommand{\ConstantTok}[1]{\textcolor[rgb]{0.00,0.00,0.00}{#1}}
\newcommand{\CharTok}[1]{\textcolor[rgb]{0.31,0.60,0.02}{#1}}
\newcommand{\SpecialCharTok}[1]{\textcolor[rgb]{0.00,0.00,0.00}{#1}}
\newcommand{\StringTok}[1]{\textcolor[rgb]{0.31,0.60,0.02}{#1}}
\newcommand{\VerbatimStringTok}[1]{\textcolor[rgb]{0.31,0.60,0.02}{#1}}
\newcommand{\SpecialStringTok}[1]{\textcolor[rgb]{0.31,0.60,0.02}{#1}}
\newcommand{\ImportTok}[1]{#1}
\newcommand{\CommentTok}[1]{\textcolor[rgb]{0.56,0.35,0.01}{\textit{#1}}}
\newcommand{\DocumentationTok}[1]{\textcolor[rgb]{0.56,0.35,0.01}{\textbf{\textit{#1}}}}
\newcommand{\AnnotationTok}[1]{\textcolor[rgb]{0.56,0.35,0.01}{\textbf{\textit{#1}}}}
\newcommand{\CommentVarTok}[1]{\textcolor[rgb]{0.56,0.35,0.01}{\textbf{\textit{#1}}}}
\newcommand{\OtherTok}[1]{\textcolor[rgb]{0.56,0.35,0.01}{#1}}
\newcommand{\FunctionTok}[1]{\textcolor[rgb]{0.00,0.00,0.00}{#1}}
\newcommand{\VariableTok}[1]{\textcolor[rgb]{0.00,0.00,0.00}{#1}}
\newcommand{\ControlFlowTok}[1]{\textcolor[rgb]{0.13,0.29,0.53}{\textbf{#1}}}
\newcommand{\OperatorTok}[1]{\textcolor[rgb]{0.81,0.36,0.00}{\textbf{#1}}}
\newcommand{\BuiltInTok}[1]{#1}
\newcommand{\ExtensionTok}[1]{#1}
\newcommand{\PreprocessorTok}[1]{\textcolor[rgb]{0.56,0.35,0.01}{\textit{#1}}}
\newcommand{\AttributeTok}[1]{\textcolor[rgb]{0.77,0.63,0.00}{#1}}
\newcommand{\RegionMarkerTok}[1]{#1}
\newcommand{\InformationTok}[1]{\textcolor[rgb]{0.56,0.35,0.01}{\textbf{\textit{#1}}}}
\newcommand{\WarningTok}[1]{\textcolor[rgb]{0.56,0.35,0.01}{\textbf{\textit{#1}}}}
\newcommand{\AlertTok}[1]{\textcolor[rgb]{0.94,0.16,0.16}{#1}}
\newcommand{\ErrorTok}[1]{\textcolor[rgb]{0.64,0.00,0.00}{\textbf{#1}}}
\newcommand{\NormalTok}[1]{#1}
\usepackage{graphicx,grffile}
\makeatletter
\def\maxwidth{\ifdim\Gin@nat@width>\linewidth\linewidth\else\Gin@nat@width\fi}
\def\maxheight{\ifdim\Gin@nat@height>\textheight\textheight\else\Gin@nat@height\fi}
\makeatother
% Scale images if necessary, so that they will not overflow the page
% margins by default, and it is still possible to overwrite the defaults
% using explicit options in \includegraphics[width, height, ...]{}
\setkeys{Gin}{width=\maxwidth,height=\maxheight,keepaspectratio}
\IfFileExists{parskip.sty}{%
\usepackage{parskip}
}{% else
\setlength{\parindent}{0pt}
\setlength{\parskip}{6pt plus 2pt minus 1pt}
}
\setlength{\emergencystretch}{3em}  % prevent overfull lines
\providecommand{\tightlist}{%
  \setlength{\itemsep}{0pt}\setlength{\parskip}{0pt}}
\setcounter{secnumdepth}{0}
% Redefines (sub)paragraphs to behave more like sections
\ifx\paragraph\undefined\else
\let\oldparagraph\paragraph
\renewcommand{\paragraph}[1]{\oldparagraph{#1}\mbox{}}
\fi
\ifx\subparagraph\undefined\else
\let\oldsubparagraph\subparagraph
\renewcommand{\subparagraph}[1]{\oldsubparagraph{#1}\mbox{}}
\fi

%%% Use protect on footnotes to avoid problems with footnotes in titles
\let\rmarkdownfootnote\footnote%
\def\footnote{\protect\rmarkdownfootnote}

%%% Change title format to be more compact
\usepackage{titling}

% Create subtitle command for use in maketitle
\newcommand{\subtitle}[1]{
  \posttitle{
    \begin{center}\large#1\end{center}
    }
}

\setlength{\droptitle}{-2em}

  \title{STA 138 Project Problem 2}
    \pretitle{\vspace{\droptitle}\centering\huge}
  \posttitle{\par}
    \author{Dandi Peng, Yuhan Ning}
    \preauthor{\centering\large\emph}
  \postauthor{\par}
      \predate{\centering\large\emph}
  \postdate{\par}
    \date{2/1/2019}


\begin{document}
\maketitle

\subsection{1. Introduction}\label{introduction}

 We are interested in how gender and death type are related in dataset
`horror'. In this project, we aim to check dependence of two variables
with assumption that all samples are random.

\subsection{2. Summary}\label{summary}

\includegraphics[width=0.5\linewidth]{problem_2_files/figure-latex/unnamed-chunk-1-1}
\includegraphics[width=0.5\linewidth]{problem_2_files/figure-latex/unnamed-chunk-1-2}
There are 251 samples in the dataset `horror' and the two-way table of
counts are shown below:

\begin{verbatim}
##         death
## gen      BFT Other Shot Stabbed
##   Female  38    28   23      61
##   Male    14     6   39      42
\end{verbatim}

 We plot barplot and mosaicplot to catch a better glimpse of the counts
in different categories. Clearly, for different death types, group
female has highest number of deaths in level `Other' and lowest number
of deaths in `Shot'. The uneven distribution of counts implies the two
factors may not be independent. Therefore, we use hypothesis tests and
confidence intervals to check further.

\subsection{3. Analysis}\label{analysis}

\textbf{Pearson's Chi-Square Test}

\(H0: P(female|BFT) = P(female|Other) = P(female|Shot) = P(female|Stabbed)\)
\(Ha:\) At least one probability in H0 is not the same.

Under the null hypothesis, \(X^2\) follows chi-square distribution with
d.f = (2-1)*(4-1) = 3

 Test statistic is 24.3067077 and p-value is
2.1554834\times 10\^{}\{-5\}.

We got table of expected value, table of standard residuals and table of
\(X^2\):

\begin{verbatim}
##         death
## gen          BFT    Other     Shot  Stabbed
##   Female 31.0757 20.31873 37.05179 61.55378
##   Male   20.9243 13.68127 24.94821 41.44622
\end{verbatim}

\begin{verbatim}
##         death
## gen             BFT      Other       Shot    Stabbed
##   Female  2.1991361  2.8891377 -4.1938149 -0.1449092
##   Male   -2.1991361 -2.8891377  4.1938149  0.1449092
\end{verbatim}

\begin{verbatim}
##         death
## gen              BFT       Other        Shot     Stabbed
##   Female 1.542876698 2.903823139 5.329104657 0.004982272
##   Male   2.291401037 4.312608622 7.914511866 0.007399413
\end{verbatim}

 Stabbed differed the least from what was expected under the null. Shot
tended to report less than expected under H0 and it contributed the most
to the rejection of H0.

\textbf{Wilson-Adjusted Bonferroni Corrected Confidence Intervals}

To know how gender depends on death types, we'd like to compare the
probability of being female for different death types using Bonferroni
corrected confidence interval.

Since we have four groups (BFT, Other, Shot, Stabbed), we compare 6
differences in total (4 chooses 2 equals 6).

\(g = 6\); \(alpha = 0.05\)

\begin{verbatim}
##                                           Lower       Upper
## P(female|BFT)-P(female|Other)     -0.3202795270  0.15361286
## P(female|BFT)-P(female|Shot)       0.1206199680  0.57382448
## P(female|BFT)-P(female|Stabbed)   -0.0729209474  0.33641301
## P(female|Other)-P(female|Shot)     0.1943894702  0.66672164
## P(female|Other)-P(female|Stabbed) -0.0001284661  0.43028720
## P(female|Shot)-P(female|Stabbed)  -0.4192395177 -0.01171286
\end{verbatim}

\subsection{4. Interpretation}\label{interpretation}

 From Pearson's Test, p-value is 2.1554834\times 10\^{}\{-5\}. If gender
and death type are independent, we would observe our data or more
extreme with probability 2.1554834\times 10\^{}\{-5\}. Since p value is
much smaller than alpha, we reject H0 and conclude that gender and death
types are dependent.

Continually, we compare 6 pairs of difference. Among the results:

\begin{enumerate}
\def\labelenumi{\arabic{enumi}.}
\item
  Confidence Intervals for
  P(female\textbar{}BFT)-P(female\textbar{}Other),
  P(female\textbar{}BFT)-P(female\textbar{}Stabbed),
  P(female\textbar{}Other)-P(female\textbar{}Stabbed) contain 0;
\item
  P(female\textbar{}BFT)-P(female\textbar{}Shot),
  P(female\textbar{}Other)-P(female\textbar{}Shot) are larger than 0;
\item
  P(female\textbar{}Shot)-P(female\textbar{}Stabbed) is smaller than 0.
\end{enumerate}

We are overall 95\% confident that the probabilities for female to die
in BFT, Other and Stabbed are the same, which is larger than the
probability of Shot. In other words, it is least probable for women to
die of shot.

\subsection{5. Conclusion}\label{conclusion}

Based on above tests and calculations, we find that gender and death
type are dependent. The probability for female to die of shot is less
than other death types. The probability of other 3 death types for
female is approximately the same.

\subsubsection{R Appendix}\label{r-appendix}

\begin{Shaded}
\begin{Highlighting}[]
\NormalTok{horror =}\StringTok{ }\KeywordTok{read.csv}\NormalTok{(}\StringTok{'horror.csv'}\NormalTok{)}
\NormalTok{n_sum =}\StringTok{ }\KeywordTok{nrow}\NormalTok{(horror)}
\NormalTok{table_data =}\StringTok{ }\KeywordTok{table}\NormalTok{(horror)}
\NormalTok{y1 =}\StringTok{ }\NormalTok{table_data[}\DecValTok{1}\NormalTok{,}\DecValTok{1}\NormalTok{]}
\NormalTok{y2 =}\StringTok{ }\NormalTok{table_data[}\DecValTok{1}\NormalTok{,}\DecValTok{2}\NormalTok{]}
\NormalTok{y3 =}\StringTok{ }\NormalTok{table_data[}\DecValTok{1}\NormalTok{,}\DecValTok{3}\NormalTok{]}
\NormalTok{y4 =}\StringTok{ }\NormalTok{table_data[}\DecValTok{1}\NormalTok{,}\DecValTok{4}\NormalTok{]}
\NormalTok{n1 =}\StringTok{ }\KeywordTok{sum}\NormalTok{(table_data[,}\DecValTok{1}\NormalTok{])}
\NormalTok{n2 =}\StringTok{ }\KeywordTok{sum}\NormalTok{(table_data[,}\DecValTok{2}\NormalTok{])}
\NormalTok{n3 =}\StringTok{ }\KeywordTok{sum}\NormalTok{(table_data[,}\DecValTok{3}\NormalTok{])}
\NormalTok{n4 =}\StringTok{ }\KeywordTok{sum}\NormalTok{(table_data[,}\DecValTok{4}\NormalTok{])}
\KeywordTok{barplot}\NormalTok{(table_data, }\DataTypeTok{beside =} \OtherTok{TRUE}\NormalTok{, }\DataTypeTok{main =} \StringTok{'Gender vs Death Type'}\NormalTok{, }\DataTypeTok{legend.text =} \KeywordTok{rownames}\NormalTok{(table_data), }\DataTypeTok{args.legend =} \KeywordTok{list}\NormalTok{(}\DataTypeTok{x =} \StringTok{'topright'}\NormalTok{)) }
\NormalTok{table2 =}\StringTok{ }\KeywordTok{table}\NormalTok{(horror}\OperatorTok{$}\NormalTok{death, horror}\OperatorTok{$}\NormalTok{gen)}
\KeywordTok{mosaicplot}\NormalTok{(table2, }\DataTypeTok{main =} \StringTok{'Gender vs Death Type'}\NormalTok{)}
\NormalTok{table_data}
\NormalTok{pearson.test =}\StringTok{ }\KeywordTok{chisq.test}\NormalTok{(table_data, }\DataTypeTok{correct =} \OtherTok{FALSE}\NormalTok{)}
\NormalTok{test.stat =}\StringTok{ }\NormalTok{pearson.test}\OperatorTok{$}\NormalTok{statistic}
\NormalTok{p.val =}\StringTok{ }\NormalTok{pearson.test}\OperatorTok{$}\NormalTok{p.value}
\NormalTok{expt =}\StringTok{ }\NormalTok{pearson.test}\OperatorTok{$}\NormalTok{expected}
\NormalTok{stdres =}\StringTok{ }\NormalTok{pearson.test}\OperatorTok{$}\NormalTok{stdres}
\NormalTok{x_sqr =}\StringTok{ }\NormalTok{(pearson.test}\OperatorTok{$}\NormalTok{observed }\OperatorTok{-}\StringTok{ }\NormalTok{pearson.test}\OperatorTok{$}\NormalTok{expected)}\OperatorTok{^}\DecValTok{2}\OperatorTok{/}\NormalTok{pearson.test}\OperatorTok{$}\NormalTok{expected}
\NormalTok{expt}
\NormalTok{stdres}
\NormalTok{x_sqr}
\NormalTok{g =}\StringTok{ }\DecValTok{6}
\NormalTok{alpha =}\StringTok{ }\FloatTok{0.05}
\NormalTok{ci1 =}\StringTok{ }\KeywordTok{prop.test}\NormalTok{(}\KeywordTok{c}\NormalTok{(y1}\OperatorTok{+}\DecValTok{1}\NormalTok{, y2}\OperatorTok{+}\DecValTok{1}\NormalTok{), }\KeywordTok{c}\NormalTok{(n1}\OperatorTok{+}\DecValTok{2}\NormalTok{, n2}\OperatorTok{+}\DecValTok{2}\NormalTok{), }\DataTypeTok{correct =} \OtherTok{FALSE}\NormalTok{, }\DataTypeTok{conf.level =} \DecValTok{1} \OperatorTok{-}\StringTok{ }\NormalTok{alpha}\OperatorTok{/}\NormalTok{g)}\OperatorTok{$}\NormalTok{conf.int[}\DecValTok{1}\OperatorTok{:}\DecValTok{2}\NormalTok{]}
\NormalTok{ci2 =}\StringTok{ }\KeywordTok{prop.test}\NormalTok{(}\KeywordTok{c}\NormalTok{(y1}\OperatorTok{+}\DecValTok{1}\NormalTok{, y3}\OperatorTok{+}\DecValTok{1}\NormalTok{), }\KeywordTok{c}\NormalTok{(n1}\OperatorTok{+}\DecValTok{2}\NormalTok{, n3}\OperatorTok{+}\DecValTok{2}\NormalTok{), }\DataTypeTok{correct =} \OtherTok{FALSE}\NormalTok{, }\DataTypeTok{conf.level =} \DecValTok{1} \OperatorTok{-}\StringTok{ }\NormalTok{alpha}\OperatorTok{/}\NormalTok{g)}\OperatorTok{$}\NormalTok{conf.int[}\DecValTok{1}\OperatorTok{:}\DecValTok{2}\NormalTok{]}
\NormalTok{ci3 =}\StringTok{ }\KeywordTok{prop.test}\NormalTok{(}\KeywordTok{c}\NormalTok{(y1}\OperatorTok{+}\DecValTok{1}\NormalTok{, y4}\OperatorTok{+}\DecValTok{1}\NormalTok{), }\KeywordTok{c}\NormalTok{(n1}\OperatorTok{+}\DecValTok{2}\NormalTok{, n4}\OperatorTok{+}\DecValTok{2}\NormalTok{), }\DataTypeTok{correct =} \OtherTok{FALSE}\NormalTok{, }\DataTypeTok{conf.level =} \DecValTok{1} \OperatorTok{-}\StringTok{ }\NormalTok{alpha}\OperatorTok{/}\NormalTok{g)}\OperatorTok{$}\NormalTok{conf.int[}\DecValTok{1}\OperatorTok{:}\DecValTok{2}\NormalTok{]}
\NormalTok{ci4 =}\StringTok{ }\KeywordTok{prop.test}\NormalTok{(}\KeywordTok{c}\NormalTok{(y2}\OperatorTok{+}\DecValTok{1}\NormalTok{, y3}\OperatorTok{+}\DecValTok{1}\NormalTok{), }\KeywordTok{c}\NormalTok{(n2}\OperatorTok{+}\DecValTok{2}\NormalTok{, n3}\OperatorTok{+}\DecValTok{2}\NormalTok{), }\DataTypeTok{correct =} \OtherTok{FALSE}\NormalTok{, }\DataTypeTok{conf.level =} \DecValTok{1} \OperatorTok{-}\StringTok{ }\NormalTok{alpha}\OperatorTok{/}\NormalTok{g)}\OperatorTok{$}\NormalTok{conf.int[}\DecValTok{1}\OperatorTok{:}\DecValTok{2}\NormalTok{]}
\NormalTok{ci5 =}\StringTok{ }\KeywordTok{prop.test}\NormalTok{(}\KeywordTok{c}\NormalTok{(y2}\OperatorTok{+}\DecValTok{1}\NormalTok{, y4}\OperatorTok{+}\DecValTok{1}\NormalTok{), }\KeywordTok{c}\NormalTok{(n2}\OperatorTok{+}\DecValTok{2}\NormalTok{, n4}\OperatorTok{+}\DecValTok{2}\NormalTok{), }\DataTypeTok{correct =} \OtherTok{FALSE}\NormalTok{, }\DataTypeTok{conf.level =} \DecValTok{1} \OperatorTok{-}\StringTok{ }\NormalTok{alpha}\OperatorTok{/}\NormalTok{g)}\OperatorTok{$}\NormalTok{conf.int[}\DecValTok{1}\OperatorTok{:}\DecValTok{2}\NormalTok{]}
\NormalTok{ci6 =}\StringTok{ }\KeywordTok{prop.test}\NormalTok{(}\KeywordTok{c}\NormalTok{(y3}\OperatorTok{+}\DecValTok{1}\NormalTok{, y4}\OperatorTok{+}\DecValTok{1}\NormalTok{), }\KeywordTok{c}\NormalTok{(n3}\OperatorTok{+}\DecValTok{2}\NormalTok{, n4}\OperatorTok{+}\DecValTok{2}\NormalTok{), }\DataTypeTok{correct =} \OtherTok{FALSE}\NormalTok{, }\DataTypeTok{conf.level =} \DecValTok{1} \OperatorTok{-}\StringTok{ }\NormalTok{alpha}\OperatorTok{/}\NormalTok{g)}\OperatorTok{$}\NormalTok{conf.int[}\DecValTok{1}\OperatorTok{:}\DecValTok{2}\NormalTok{]}
\NormalTok{results =}\StringTok{ }\KeywordTok{rbind}\NormalTok{(ci1, ci2, ci3, ci4, ci5, ci6)}
\KeywordTok{colnames}\NormalTok{(results) =}\StringTok{ }\KeywordTok{c}\NormalTok{(}\StringTok{'Lower'}\NormalTok{,}\StringTok{'Upper'}\NormalTok{)}
\KeywordTok{rownames}\NormalTok{(results) =}\StringTok{ }\KeywordTok{c}\NormalTok{(}\StringTok{'P(female|BFT)-P(female|Other)'}\NormalTok{,}\StringTok{'P(female|BFT)-P(female|Shot)'}\NormalTok{,}\StringTok{'P(female|BFT)-P(female|Stabbed)'}\NormalTok{,}\StringTok{'P(female|Other)-P(female|Shot)'}\NormalTok{,}\StringTok{'P(female|Other)-P(female|Stabbed)'}\NormalTok{,}\StringTok{'P(female|Shot)-P(female|Stabbed)'}\NormalTok{)}
\NormalTok{results}
\end{Highlighting}
\end{Shaded}


\end{document}
